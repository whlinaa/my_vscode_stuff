\documentclass[letterpaper,11pt,addpoints]{exam}
% \usepackage{examStyle}
\usepackage{/Users/whlin/Library/CloudStorage/OneDrive-HKUSTConnect/Documents/Latex/Styles/My_Style/examStyle}
\usepackage[top=0.3in, bottom=0.7in, left=0.3in, right=0.4in]{geometry}

\newcommand{\class}{SEHH 2241}
\newcommand{\term}{Fall 2022}
\newcommand{\examnum}{midterm test}
% \newcommand{\examdate}{1/1/2014}
% \newcommand{\timelimit}{60 Minutes}

% \pagestyle{headandfoot}
\pagestyle{foot}
% \firstpageheader{}{}{}
% \firstpagefooter{}{Page \thepage\ of \numpages}{}
\firstpagefooter{}{\thepage\//\numpages}{}
% \runningheader{\class}{\examnum}{\examdate}
% \runningheadrule
% \runningfooter{}{Page \thepage\ of \numpages}{}
\runningfooter{}{\thepage\//\numpages}{}


\setlength\linefillheight{.25in} % controls the line spacing of \fillwithlines. defaults to .25in

% for automatic numbering of tables
\newcounter{magicrownumbers}
\newcommand\rownumber{\stepcounter{magicrownumbers}\arabic{magicrownumbers}}

\begin{document}

\noindent
\begin{tabular*}{\textwidth}{l @{\extracolsep{\fill}} r @{\extracolsep{6pt}} l}
\class & Name (English): & \makebox[2in]{\hrulefill}\\
\term &Student ID: & \makebox[2in]{\hrulefill}\\
\examnum &Class: & \makebox[2in]{\hrulefill}\\
% \textbf{\examdate} &&\\
% \textbf{Time Limit: \timelimit} &&
% \textbf{\c{Due time: \dueTime}} &&
\end{tabular*}\\
\rule[2ex]{\textwidth}{2pt}

% \subsection*{Before attempting the assignment, please read the following carefully.}
\begin{itemize}
\item Time allowed: 75 minutes
\item IMPORTANT: In all the questions, the symbol $R$ denote the last digit of your student ID.
\item This assessment contains \numpages\ pages (including this cover page) and \numquestions\ questions.
\item You must write your answers using the notations introduced in class.
\item Show all your steps clearly for all questions. 
\item All your answers should be exact. If exact values cannot be obtained, round your final answers to four decimal places.
% \item All your answers must be written in the the space provided.
\item Write your answers neatly. Unclear answers will not be marked.
\end{itemize}

% \begin{center}
% % Grade Table (for teacher use only)\\
% \addpoints
% \gradetable[v][questions]
% \end{center}



\newpage
\begin{questions}
\question[20]~

\begin{parts}
\part Write a truth table for the statement below. Then, determine whether the statement is a tautology, contradiction, or contingency. NOTE: for the truth table, you need to write at least the three main connective columns. 
\begin{tabular}{c|c|c|c}
$p$ &$q$& $r$ &$[(\sim r \wedge q) \leftrightarrow \sim p] \oplus \neg[r \rightarrow(p \vee \sim q)]$\\
\hline
T&T&T&\\
\hline
T&T&F&\\
\hline
T&F&T&\\
\hline
T&F&F&\\
\hline
F&T&T&\\
\hline
F&T&F&\\
\hline
F&F&T&\\
\hline
F&F&F&\\
\end{tabular}
\part Are the following statement true? If it is true, prove by logical rules. If it is false, give one set of specific truth values of the variables to show it.
\[p\wedge(q\oplus r)\equiv(p\wedge q)\oplus(p\wedge r)\]
\part (version B) Are the following statement true? If it is true, prove by logical rules. It it is false, give one set of specific truth values of the variables to show it.
\[p\vee(q\oplus r)\equiv(p\vee q)\oplus(p\vee r)\]
\end{parts}

\question[15]~
\begin{parts}
% Q10, ex3.7, CZ
\part Does there exists two real numbers $a>R$ and $b>R$ such that $\sqrt{a+R}+\sqrt{b+R} = \sqrt{a+b+2R}$? If so, find $a$ and $b$. If not, prove that such two real numbers do not exist.
% \part Prove or disprove that $\sqrt{2}+\sqrt{+R}$ is an irrational number or rational number, You may take for granted that $\sqrt{2}$ is irrational, since this has been proved in class. 
\end{parts}

\question[24]~
\begin{parts}
% \part  Let $A, B,$ and $C$ be sets. Prove or disprove the following statement: \[A-(B \cup C)=(A-B) \cap(A-C).\]
% If you think the statement is true, prove by logical rules. If you think it is false, give a specific example of the sets $A, B,$ and $C$.

\part Let $U=\{1,2,3,4,5,6,7\}, A=\{1,2,4,5\},B=\{1,3,4,6\}$, where $U$ denotes the universal set. Find the following. In this question, you need to write down the final answers only. \textbf{No steps are required for this question. }
\begin{subparts}
  % \subpart $|\{\lfloor\frac{x}{2}\rfloor-1\mid x\in\R, ~0\le x<4\}|$
  \subpart $|\overline{B}\times (A-B)|$
  \subpart $P( \overline{ A\cup B})$
  % \subpart $|\overline{B}\times (A-B)|$
  \subpart $|B\times(\overline{A\cap B})|$
  \subpart $\overline{A}\cap(B\cup\{\emptyset^{R+1}\})$
  \subpart $|P( A\times\overline{B}\times A)|$
  \subpart $|\{a+b+R\mid a\in A, b\in B\}|$
  \subpart $\{1\}\times\{2\}\times\{2\}\times\{1\}\times\{R\}\times\{R^2\}\times\emptyset$
  % \subpart $\left(\{1\}\times\{2\}\right)\times\left(\{2\}\times\{1\}\right)$
  % \subpart $\{1\}^2\times \{2\}$
  \subpart $|\{\emptyset\}^{R+1}|$
  % \subpart $\{\}^{3} \cup \{\emptyset\}^2$
\end{subparts}
\end{parts}

\question[16]~
\begin{parts}
% \part Prove by mathematical induction that $n^4+2n^3-n^2+14n$ is divisible by 8, where $n\in\Z^+$.
% \part (version B) Prove by mathematical induction that $(5n+24)6^n+1$ is divisible by 25, where $n\in\Z^+$.
\part Prove by mathematical induction that $2^{4(n+R)-1}+3^{3(n+R)-2}$ is divisible by 11, for all $n\in\Z^+$.
\end{parts}

\question[25]~
For each of the sub-questions below, explain your answers clearly. 
\begin{parts}
\part A function $f: \mathbb{Z} \rightarrow \mathbb{Z}$ is defined by $f(x)=3 x+5 .$ 
\part Let $f: \mathbb{N} \rightarrow \mathbb{N}$ be defined by $f(x)=\lfloor 2 x-2 + (R+2)^2 \rfloor-\lceil x-3 + (R+2)^2\rceil$
\begin{subparts}
  \subpart Is $f$ one-to-one?
  \subpart Is $f$ onto?
\end{subparts}
\part Let $f: \R\times\R\rightarrow\R\times\R$ be defined by $f(x,y)=(3y+(R+2)^2,-5x^3)$.
\begin{subparts}
  \subpart Is $f$ one-to-one?
  \subpart Is $f$ onto?
\end{subparts}
\end{parts}

\end{questions}





\end{document}
