% good commands:
% \makeemptybox{5in}
% \makeemptybox{\fill}
% \fillwithlines{3in}
% \fillwithlines{\fill}
% vspace*{3in} # You can also use the \vspace command, the difference being that space inserted by \vspace will be deleted if it occurs at the top of a new page, whereas space inserted by \vspace* will never be deleted.
% vspace*{\stretch{1}}
% If you want to leave all the remaining space on the page blank, you should give the commands
% \vspace*{\stretch{1}}
% \newpage

% If you want to equally distribute the blank space among several questions on the page, then just put \vspace*{\stretch{1}} after each of the questions and use \newpage to end the page. You can also distribute the available space in some other ratio. For example, to give one of the questions on the page twice as much space as any of the others, put \vspace*{\stretch{2}} after that question and \vspace*{\stretch{1}} after each of the others.

\documentclass[letterpaper,10pt,addpoints]{exam}


% \usepackage{examStyle}
\usepackage{/Users/whlin/Library/CloudStorage/OneDrive-HKUSTConnect/Documents/Latex/Styles/My_Style/examStyle}
\usepackage[top=0.3in, bottom=0.7in, left=0.3in, right=0.4in]{geometry}


\usepackage{array}
\newcolumntype{P}[1]{>{\centering\arraybackslash}p{#1}}


\newcommand{\class}{SEHH 2311}
\newcommand{\term}{Spring 2023}
\newcommand{\examnum}{midterm}
% \newcommand{\examdate}{1/1/2014}
% \newcommand{\dueTime}{23 Mar 5:00pm}
% \newcommand{\timelimit}{60 Minutes}

% \pagestyle{headandfoot}
\pagestyle{foot}
% \firstpageheader{}{}{}
% \firstpagefooter{}{Page \thepage\ of \numpages}{}
\firstpagefooter{}{\thepage\//\numpages}{}
% \runningheader{\class}{\examnum}{\examdate}
% \runningheadrule
% \runningfooter{}{Page \thepage\ of \numpages}{}
\runningfooter{}{\thepage\//\numpages}{}


\setlength\linefillheight{.25in} % controls the line spacing of \fillwithlines. defaults to .25in

% for automatic numbering of tables
\newcounter{magicrownumbers}
\newcommand\rownumber{\stepcounter{magicrownumbers}\arabic{magicrownumbers}}

% https://tex.stackexchange.com/questions/351108/how-to-continue-lines-on-next-page-in-exam-class-fillwithlinesw
\makeatletter
\def\fillwithlines#1{%
  \begingroup
  \ifhmode
    \par
  \fi
  \hrule height \z@
  \nobreak
  \setbox0=\hbox to \hsize{\hskip \@totalleftmargin
          \vrule height \linefillheight depth \z@ width \z@
          \linefill}%
  % We use \cleaders (rather than \leaders) so that a given
  % vertical space will always produce the same number of lines
  % no matter where on the page it happens to start:
  \dimen0=\ht0
  \loop\ifdim\dimen0<#1\relax
    \advance\dimen0 by \ht0
    \copy0\space
  \repeat
  \endgroup
}
\makeatother


\begin{document}

\noindent
\begin{tabular*}{\textwidth}{l @{\extracolsep{\fill}} r @{\extracolsep{6pt}} l}
\class & Name (English): & \makebox[2in]{\hrulefill}\\
\term &Student ID: & \makebox[2in]{\hrulefill}\\
\examnum &Class: & \makebox[2in]{\hrulefill}\\
% \textbf{\examdate} &&\\
% \textbf{Time Limit: \timelimit} &&
% \textbf{\c{Due time: \dueTime}} &&
\end{tabular*}\\
\rule[2ex]{\textwidth}{2pt}

% \subsection*{Before attempting the assignment, please read the following carefully.}
\begin{itemize}
% \item This assessment contains \numpages\ pages (including this cover page) and \numquestions\ questions.
\item All your answers must be written in the the space provided.
\item You must write your answers using the notations introduced in class.
\item Show all your steps clearly for all questions. 
% \item All your answers should be exact. If exact values cannot be obtained, round your final answers to four decimal places.
\item Write your answers neatly. Unclear answers will not be marked.
\item You may use the facts and theorems introduced in lectures and tutorials. All other facts/theorems that are not covered must be rigorously proved before being used. 
\item If necessary, round your final answer of each part of the question to 4 decimal places.
\end{itemize}

% \subsection*{Submission guidelines}
% \begin{itemize}
% \item DO NOT convert this .pdf file to any other file format, such as .word.
% \item Submit your work to Moodle. Email submission is NOT accepted. 
% \item \underline{SCAN} the ENTIRE file (including the front page) and saved it as ONE single PDF file. That is, even if you did not attempt some questions, you still need to scan that empty pages. Also, do not scan a portion of a page only---we need the ENTIRE page. 
% \item Scan the file IN THE ORDER OF THE PAGES. That is, page 1 should come before page 2, and page 2 should come before page 3 and so on. 
% \item Make sure that your work is properly scanned. Over-sized, blurred or upside-down pages will NOT be graded.
% \item If you do not have access to a scanner, you may use a mobile app (e.g., CamScanner) to scan it. The scanning by a physical scanner is highly preferable, though.  
% % \item You may use a tablet (e.g., iPad) to write down your answers. However, you must ensure that your file can be opened using Adobe Reader on a Windows machine. If the file cannot be opened, no marks will be given. 
% %In particular, \textbf{do not use the software \underline{Onenote} to do the assignment, because there will be difficulty printing out your assignment. }
% % \item You may type your solutions using a computer.

% \item Upload the PDF and double check the integrity of your file. Occasionally the scanned file may appear differently on the Moodle system.
% \item Name your file by your full English name. For example, if your name is Chan Tai Man, then your file name is Chan Tai Man.
% \item Late submission, regardless of the reasons, will not be accepted. Submit your work to Moodle some time ahead of the deadline. Late submissions due to slow internet speed, for instance, will not be accepted.
% \item The maximum submission size is 20MB.
% % \item The maximum submission size is 20MB. A well-scanned submission should have a size well below that limit. 
% \end{itemize}



% \begin{center}
% % Grade Table (for teacher use only)\\
% \addpoints
% \gradetable[v][questions]
% \end{center}

% \noindent
% \rule[2ex]{\textwidth}{2pt}

\newpage
\begin{questions}

% \question[10]
%   A discrete random variable $X$ has the following distribution function :
%   $$
%   \mathrm{F}(x)=\left\{\begin{array}{llr}
%   0 & \text { for } & x<-2 \\
%   0.2 & \text { for } & -2 \leq x<0 \\
%   0.7 & \text { for } & 0 \leq x<1 \\
%   1 & \text { for } & x \geq 1
%   \end{array}\right.
%   $$
%   \begin{parts}
%     \part Write down the probability mass function of $X$.
%     % \part Find $Var(X)$.
%     % \part Let $Y=Z^2+X^2+2 X+1$, where $Z$ follows a standard normal distribution. Find $E(Y)$.
%     \part Let $Y=X^2+2 X+1$. Find $E(Y)$.
%   \end{parts}
\question[40]
On a certain day, Peter travels by train from Station $A$ to Station $C$ via Station B. He boards a train at Station $A$ and the train leaves for Station $B$ at 9:00 a.m. The time $X$ (in minutes) taken by the train to reach Station $B$ has a probability density function given by:
$$
f(x)= \begin{cases}\frac{1}{15} & \text { for } \quad k<x \leq 66, \\ \frac{13}{60}-\frac{1}{360} x & \text { for } \quad 66<x<78, \\ 0 & \text { otherwise. }\end{cases}
$$
\begin{parts}
  \part Show that $k=54$ and find $E(X)$.
  \part Find $Var(X)$.
  \part Find the cdf of $X$.
  \part Peter gets off the train at Station $B$ and boards the earliest available train for Station $C$. It is known that the following trains leave Station $B$ for Station $C$ on that day:
  $$
  \begin{array}{cc}
  \text { Train number } & \text { Departure time } \\
  T_1 & 10: 06 \text { a.m. } \\
  T_2 & 10: 24 \text { a.m. }
  \end{array}
  $$
  Assume that the time it takes to walk from one train to another at Station $B$ is negligible. Suppose that each train takes $30.4$ minutes to travel from Station $B$ to Station $C$.
  \begin{subparts}
    \subpart Find the probability that Peter will get on Train $T_1$.
    \subpart Find Peter's expected arrival time at Station $C$.
    \subpart Find the expected length of time (in minutes) Peter has to stay at Station $B$.
  \end{subparts}
  \part Suppose that instead of leaving for Station $B$ at 9:00am, the train at Station $A$ now leaves for Station $B$ at 9:10am. Find the probabiity that Peter reaches Station $B$ by 10:20am.
\end{parts}

% \question[15]
% A certain change in the production process of a certain factory is being considered. Samples were taken using both the existing and the new procedures so as to determine if the new process results in a significant improvement.
% \begin{parts}
%   \part Based on the following sample data to find a $90 \%$ confidence interval for the true difference in the fraction of defectives between the existing and the new process.

%   \begin{tabular}{|l|c|c|}
%     \hline & Sample Size & No. of Defective Found \\
%     \hline Existing Process & 1,500 & 75 \\
%     \hline New Process & 2,000 & 80 \\
%     \hline
%     \end{tabular}

%   \part Based on the results from part (a), does the new process results in a significant improvement? Explain.
% \end{parts}

\question[40]
A research study was conducted to estimate the difference in the amount of chemical measured at two different stations along the Shing Mun River in Shatin. Fifteen samples were collected from Station 1 and 12 samples were obtained from Station 2.

The results of Station 1 are $\bar{x}_1=3.84 \mathrm{mg} / \text {liter }$ and ~$s_1=3.07 \mathrm{mg} / \text {liter} $

The results of Station 2 are $\bar{x}_2=1.49 \mathrm{mg} / \text {liter }$ and ~$s_2=1.50 \mathrm{mg} / \mathrm{liter}$
\begin{parts}
  \part Construct a $90 \%$ confidence interval for the standard deviation of the measured chemical level from Station 2. State any necessary assumption.
  
  \part Construct a $98 \%$ confidence interval for the ratio between the variances of the measured chemical levels from Station 1 and Station 2.
  \part Based on the results from (b), develop a 95\% confidence interval for the difference between the average chemical levels of these two stations. What conclusion can be drawn from the confidence interval?
  \part What is the least additional samples needed to select if the company wishes to be $98 \%$ confident that the estimate of mean chemical levels from Station 1 will be within $1.0 \mathrm{mg} /$ liter of the true population mean?
\end{parts}


\question[20]
To estimate the perimeter $p$ of a circular ring, its diameter $\mu$ is measured ten times, and the sample mean is 12.5. It is known that these measurements can be modelled by a normal distribution with mean $\mu \mathrm{cm}$ and standard deviation $0.01 \mathrm{~cm}$.

\begin{parts}
  \part Find a $95 \%$ confidence interval for $\mu$.
  \part The perimeter $p$ is to be estimated by
  $$
  \hat{p}=\pi \overline{x},
  $$  
  where $\bar{x}$ is the sample mean of the above ten measurements.

  Find a $95 \%$ confidence interval for $p$.
% \begin{subparts}
%   \subpart Find the standard deviation of $\hat{p}$.
  % \subpart Find a $95 \%$ confidence interval for $p$.
% \end{subparts}
\part An alternative method is to directly measure the perimeter $p$ of the ring ten times. It is assumed that each measurement is normally distributed with mean $p \mathrm{~cm}$ and standard deviation $0.02 \mathrm{~cm}$. 

In order to attain a $0.99$ probability that the estimate of $p$ lies within $0.01 \mathrm{~cm}$ of the true value, how many additional measurements must be made?
\end{parts}

\end{questions}





\end{document}
