% Professor James Worrell : war-well % 


\documentclass[12pt, landscape]{article}   
\usepackage[landscape]{geometry}
\usepackage{ graphicx, fullpage, amssymb, color, soul, enumitem, float, verbatim, multicol}
\geometry{letterpaper}

\def \c {\textcolor{red}} 
\def \i {\textit}
\def \b {\textbf}
\def \R {\Rightarrow}

\parindent 0em

\usepackage{ifthen}



\ifthenelse{\lengthtest { \paperwidth = 11in}}
	{ \geometry{top=.5in,left=.5in,right=.5in,bottom=.5in} }
	{\ifthenelse{ \lengthtest{ \paperwidth = 297mm}}
		{\geometry{top=1cm,left=1cm,right=1cm,bottom=1cm} }
		{\geometry{top=1cm,left=1cm,right=1cm,bottom=1cm} }
	}

\begin{document}
\begin{multicols}{2}

\section{General Q}

\subsection{Intro to PhD}
Hello everyone, my name is Wing-ho Lin, and I do have an English name, which is Joe, so you may just call me Joe.\\

And I've just finished my MPhil degree in Computer Science at HKUST. And my research areas are data mining and databases. \\

But more generally, I like statistics, math, as well as computer science, of course.\\

I am also a graduate of HKCC, and I mainly studied Statistics when I was a student of this college. And I always feel fortunately to be able to study in HKCC, because I met so many wonderful teachers and they really help me a lot even after I have graduated for many years. I also made many good friends here, and I am still in contact with many of them. And this is also one of the reasons why I'd like to be a visiting lecturer here. \\

And let's talk about what I do in my spare time. I like reading news, especially Technology news. This is a good way for me to learn new ideas. \\



\subsection{Intro for PhD at Oxford}

Hello Prof Dan, My name is Joe Wing Ho Lin, you can just call me Joe. My favourite academic subjects are machine learning and statistics. I enjoy doing research on graph problems that are too computationally intensive to be solved exactly. And I like to design approximate randomized algorithms  that provide performance guarantee to this kind of graph problems.\\

And I am going to graduate from HKUST next year. At HKUST I mainly studied computer science courses, but before I came there, I actually studied mainly statistics at another college in Hong Kong for two years, and I gained my mathematics foundation here.\\

And let's talk about what I do in my spare time. I like reading news, especially Technology news. This is a good way for me to learn new ideas.\\

\subsection{What you should be chosen}

 First of all I think my academic performance is satisfactory, and I was ranked first in my Computer Science program when I was an undergraduate. And so, I should have sufficient background knowledge to conduct research in computer science.\\
 
 Second, I have relevant research experience. I'm doing data mining research right now, and I have sufficient background in statistics and probability, which are obviously important if I am going to do research on mac, or other related research topics as well. 
 \subsection{Why do a PhD}
 First of all I like doing research, and I love solving challenging problems. And I also want to have contribution to the areas that I know. And I find doing a PhD to be a very good way to improve myself. I find myself to be more critical and self-disciplined after being an MPhil student. And these qualities are going to be very useful in my entire life. 
 \subsection{Programming Skills? Your largest project?}
 
Yes, I am quite familiar with C++ because all of my previous programming courses are based on this language. I also use Matlab and R. My machine learning courses all used Matlab, and I also used Matlab for my MPhil research, because I need to plot graphs for my experiments and I also Matlab to generate synthetic datasets, for example. And so I use Matlab quite often. And I used R when I took statistics course. And I am also good at Latex, which I use to typeset everything for my assignments, my notes, and my research draft.
 \begin{enumerate}
 	\item   The largest programming project I've ever done is about making a movie ticket booking system. It provides an interface for the users so that they can do all the things you can do for a real ticket booking system.
 	\item  It consists of many lines of codes, perhaps 10 thousands. But this is a group project of two people, and it accounts for more than 50\% of my course grade. 
 	\item And, when I studied machine learning, there was a project about handwritten digits recognition that required us to implement the multilayer perceptron algorithms and  Principal component analysis algorithms in C++, not matlab. After implementation, we then need to  use 30,000 handwritten digits as the training set to training our model, and then  use 10,000 digits images as testing set.
 \end{enumerate}

 
\subsection{Introduce yourself for Master}
Hello, My name is Lin Wing Ho, Joe, you can simply call me Joe. My favourite academic subjects are machine learning and programming. \\

And now I am going to graduate from HKUST this summer. At UST I mainly studied computer science courses, but before I joined HKUST, I actually studied mainly statistics at another college in Hong Kong for two years, and I gained my mathematics foundation here.\\

I am now doing my final year project, which is a research project about using machine learning techniques to automatically detect the hidden topics of a large set of documents. For example, if we have some documents talking about animals, some are talking about education, and finance, then our system should divide the documents into 3 groups, one group is all about animals, another is education, and last one is finance.  So, this is the aim of my Final year project.\\

I am determined to further pursue my masters' degree upon graduation. And in fact, I am also determined to pursue a PhD afterwards, researching in machine learning. My career goal after obtaining my PhD degree is to first of all pick up a post-doc position at a university for around 2 years, in order to gather more research experience. And then, I'll apply for a research-related position in the job market. The companies that I'd like to work for include Microsoft, Google, Alibaba, Baidu, Amazon and so on. Well, they need a lot of machine learning researchers.\\

And now I'd like to talk about what I like to do during my spare time. I like reading news, especially Technology news. This is a good way for me to learn new ideas.\\

And I also like doing physical exercise, particularly jogging. And I actually go jogging around three times per week, and run for saround 1 hour each time. Jogging is excellent for me to relax and it is the time for me to think clearly.

\subsection{Why do you like machine learning?}
I like machine learning because it is very useful in practice. Many recent technological advances are driven by machine learning research, such as image recognision, voice recognision, and robotic control. Well they all rely on machine learning. So, I really think that this is a very promising research area to choose. And machine learning is also very statistical, and as I have studied a lot of statistics before, I can now apply what I have learnt in machine learning.

\subsection{Why do you like programming?}
Programming is really the heart of computer science. I like to create new things, I am to be a producer of a product, but not just the consumer. And I can use programming technique to create something that may not be already available in the market. For example, if I have a brillant idea, I'll have to know programming to implement it. And so, programming allows people a lot of freedom of how to create new things, and that's why I like it.

\subsection{What programming languages you know?}
I am most familiar with C++, and all of my programming courses are based on this langauge. And I am also good at R, a programming language for statistical data analysis. Three of my statistics courses are based on this langauge. And I also know Matlab, which is used in my machine learning course. And I am also good at Latex, which I used to typeset my homeworks and reports. \\


\subsection{Tell me a project you did in Matlab, R}
There was a project about handwritten digits recognition in my machine learning course, that required us to implement the multilayer perceptron algorithms and Principal component analysis algorithms. After implementation, we then need to use 30,000 handwritten digits as the training set to training our model, and then use 10,000 digits images as testing set.

\subsection{Why did you choose UK for studying, not US?}
I choose the UK because UK has an very good reputation for academic excelleance. And staying here I can learn the most authetic English. I really want to further improve my English. 


\subsection{Why don't you stay in the UST for Master's degree?}
UST is excellent, but I just want to have more overseas experience. I have always been in HK, I really want to know more about the rest of the world.


\subsection{What's Computer Science?}
For me, computer science is the study of computers and how to use computers to fulfill our tasks, some of which may not be possible to be accmomplished by a human.


\subsection{Why do you choose Oxford's Msc program?}
I liked this program because it strikes a very good balance between theory and practice. Quite frankly, the Msc programs offered by other universities are usually quite practical, this may not be in my best interest because my aim is to further pursue a PhD. So, I'll definitely need the theorical foundation to help me succeed in my future research. \\

And another plus of Oxford's Msc program is that it has a substantial research project component. What really impresses me is the fact that many of your student's dissertations have turned into famous scientific conferences. That's not common in other universities.


\subsection{Which professor you would like to be your supervisor?}
I like to be supervised by Nando de Freitas, but his academic interest is machine learning and statistics. That's a good fit for me.
% or michael Wooldridge

\subsection{Did you participate in any extra-curricula activities?}
Yes. I am currently taking part in KDD-cup, which is an international data science competition for both practioners and researchers. KDD stands for Knowledge discovery and data mining. In this competition, we are given a large dataset, for example in this year we'll be given a dataset about students participating in Massive Open Online Courses (MOOC), and we need to predict whether or not a student will drop out a course in 10 days based on his record.


\subsection{ Your secondary school results are poor. Why so?}
Yes. I wasn't a hard-working student when I was in my secondary school. And my academic results were very poor at that time.
But now I am proud to say that I have learnt a valuable lesson from this mistake and since then I totally changed myself. I am now really eager to learn, and I have been working extra hard in order to stand out. Well, I believe everyone will make mistakes in their life but the key question whether we can learn from this mistake and become a better person afterwards. So, I think making mistakes is fine as long as we learn it and do something about it.

\subsection{Why don't you choose Cambridge's program?}
The MPhil program is definitely excellent, but the course is relatively short compared with the Msc of Oxford.

\subsection{Why should Oxford choose you?}
I think I am qualified for this Msc course because of my academic performance. I have sufficient knowledge in both computer science and mathematics. And I also have experience in conducting research projects in both machine learning and data mining. These factors together make me confident that I can do well in this Master's program.


\subsection{ Your FYP?}
My final year project is about using topic modelling to automatically detect the hidden topics of a popular Internet forum in Amazon, Amazon Discussions Feedback Forum. Say for example, if people discuss topics like delivery, complaint, price here, then the algorithm will divide the posts into three groups, one is about delivery, one is complaint, and the final one is about price. And our final product is a website, that allows Amazon users to browse the Amazon Forum by topic. \\
% People discuss many different topics here, such as delivery, complaints, after-sales services, price. 

We use both Latent Dircilet allocation, or LDA and Hierachical Latent tree analysis, or HLTA as our topic model algorithms to identify the hidden topics people discuss there. But FYP isn't ended after finished detecting the topics. we still need to use another algorithm to link the posts of the forum to their corresponding topics that we have identified using topic model algorithms. \\

The challenging part is parameter tuning. There are many parameters in LDA, such as the \# of topics, the termination criteria of LDA, the \# of maximum iteration.  

\subsection{Why do you choose graduate program?}
That's because I really want to do research in machine learning, particularly in probabilistic topic model algorithms. I really see the potential of these algorithms, and so I decided to further my studies in order to obtain the necessary knowledge to help me conduct research in this area. 



\section{Programming Q}
\subsection{What's functional programming?}
functional programming is a programming paradigm where computational units are treated as a mathematical functions.

\subsection{What's object-oriented programming?}
object-oriented programming is a programming paradigm where variables are treated as obejcts. And each object has its own attributes. And each object has the procedures that they can perform. \\

Ex: C++, Java\\

Pros: \\

Key distinction between Procedural and OO: Each object in object-oriented programming has its own data structure, or attributes.

\subsection{What's procedural  programming?}
procedural programming is a programming paradigm where the building block is procedures, also functions. And these procedures are called during the execution of the program.\\

Ex.: C, Pascal
\subsection{Differences between Functional and Procedural programming?}
Functional is that you give me inputs, I give you output. But Procedural is just that the language execute the code one line by one line. 



\section{Algorithms Q}

\subsection{What's dynamic programming?}
This is a method to slove a problem by reusing the subproblems that we have already solved. We define a class of subproblems, and we can find a formula linking the current problem to the class of subproblems. Say for eg, \\

Note: memoization is a top-down approach to dynamic programming. it uses recursion, but also uses a table to record the solutions to subproblems. The key idea is to avoid solving an overlapping subproblem again. So, it a table entry is filled, we don't recurse.

\subsection{What's greedy algorithm?}
Greedy algorithms are a type of algorithm that is usually used for optimization problems. And it follows some heuristic to solve a problem. We incrementally build a solution, and at each step choose what seems to be best at the moment. The running time of such kind of algorithm is genearlly short but the solutions might not be the optimal solution sometimes.

\subsection{What can DFS/BFS solve?}
\begin{itemize}
\item Is vertex t reachable from vertex s? (just run BFS/DFS from s. check if t is mark ``discovered")
\item Is undirected graph connected?
\item Is a graph acyclic?
\end{itemize}

\subsection{sorting}
\begin{itemize}
\item Insertion sort: 
\item
\item Merge sort: divide the array into 2 subarray. Then, for each subarray, we again divide it into 2 parts, until one part has one element only. In this case, we return because an array with one element is already sorted. By using this strategy, we have two sorted arraies. Then, we need a merge algorithm to combine the two sorted arraies.  Note that need $\Theta (nlogn)$ and $\Theta (n)$ extra space for the third array.
\item Quick sort: First, we need to pick a real value for partitioning our array into two part, this value is called pivot. Then, we use an algorithm that can parition the array into two parts, one part has all the elements with values less than or equal to the value of pivot. And another one has has all the elements with values larger than the values of the pivot. And then we apply the same procedure for the two arrays. And we stop when an array has only one element because it's sorted.

Running time: $\Theta (nlogn)$ on average. $\Theta (n^2)$ worse case
\item
\item
\item
\item
\end{itemize}



\section{Computation Q}
\subsection{What's  NP-complete?}
\begin{itemize}
\item P: all decision problems that can be solved in polynomial time.
\item NP: all decision problems that can be verified in polynomial time.
\item We assume P $\neq$ NP
\item NP-Complete: a problem is NP-complete if it is a hardest problem in NP.
\item NP-complete property: If P $\neq$ NP, then any NP-complete problem cannot be solved in polynomial time (because some NP problems can't be solved in polynomial time)
\item Satisfiability(SAT): Given a boolean formula, decide whether there exists an assignment of variables to make the formula true. This is NP-complete.
\item Halting problem: Decide if a computer program always halts. It's shown that it cannot be solved by any algorithms. So, it is undecidable.
\end{itemize}
\subsection{What's deterministic/non-deterministic finite automata(DFA)?}
DFA is a machine that receives a a finite string of symbol as inputs, and the output is to either accepts or rejects this string. 
a NFA jumps determinstically from a state to another by following the transition function. 
\subsection{What's turing machine?}
Turing machine is proposed by Alan Turing. It's very much like a finite state machine but with an unlimited memory.

\section{My Questions}
\subsection{Can I audit courses offered by other departments, such as the maths department?}
\subsection{whether the computer science department of Oxford employs language tutors to help students with dissertation writing, or presentation rehearsal?}
\subsection{Do I have the freedom to choose my own supervisor, as I like to be with a }
\subsection{Can we take more courses than the course requires us to?}
\subsection{}

%\includegraphics[scale=0.35]{/Users/Ho/Desktop/ScreenShot/Oxford.png}


\section{New Q}

\subsection{Why do you want to study Msc?}
I like to pursue a Master's degree because I really enjoy studying, and my goal is to take on a research position, either in industry or in academia, after finishing my PhD. And so, having a Master's degree will be the first step towards my goal.

\subsection{Why do you want to study PhD?}

\subsection{why your software eng result is poor?}
Well it's because 60\% of the course grade was based on a project. The professor wanted us to come up with an innovative idea, but finally he might not think that our project is very innovative, and so we were not given an excellent grade in this course.

\subsection{What's covered in your SE course?}
Software development processes (such as waterfall, Agile, code and fix), writing software development requirements, design principles and patterns (program to interface, not implementation), and various software testing techniques, such as white-box testing, black-box testing, automatic testing, and so on.

\subsection{why you switched from statistics to computer science?}
Well, first of all I think statistics is an interesting and excellent subject to study. And I absolutely love it. But my favourite area in statistics is actually data analysis, which requires a lot of programming and algorithm skills. And that's why I switched from studying statistics to computer science, and now I usually take courses that are related to data analysis, such as machine learning, data mining, and bayesian networks.

\subsection{why do you like computer science?}
Studying computer science allows me to create new things, say new apps, new software, but not just the consumer of products made by other people. 

And also, knowing computer science well means that I can study other subjects a lot easier, such as statistics, which also requires programming and algorithm skills 

\subsection{What's concurrency?}
It means a computer can support multiple tasks making progress. It's achieved by allowing them to use the processor at different short time instance. 

\subsection{What's parallelism?}
it means a computer can perform mutiple tasks at the same time. It can be supported only by computers with more than one processor.

\subsection{important definitions}
\begin{itemize}
\item Tree: An acyclic, connected and undirected graph
\item Forest: An acyclic and undirected graph 
\item DAG: Directed, acyclic graph
\item Adjacency matrix: method to represent a graph. Good for dense graph
\item Adjacency list: method to represent a graph. Good for sparse graph
\item
\item
\item
\end{itemize}

\subsection{Clustering}
\begin{enumerate}
	\item disadvantage of K-means: (1) need to specify K, (2) problems with initial centroid assignments. 
	\item
	\item
	\item
	\item
	\item
\end{enumerate}

\subsection{EM algorithm}
\begin{enumerate}
	\item given a dataset such that each data point is given without their corresponding label. 
	\item assume that there are several probability distributions so that each data point was generated by one of the distributions. 
	\item Make initial guess on the proportion of each class.
	\item E step is about calculating the partial assignment probability of each data. Such probability is exactly the \textbf{expected value} of the data belonging to a particular class. In theory, the complete data includes $k$ indicator random variables, each of which is equal to 1 if the data is from that class. So, the expected value of that indicator is exactly the probability. Note that in this step, we need the proportion of each class in order to calculate the probability and that's why we need to make initial guess on the proportions.
	\item M step is about recalculating the proportion of each class using the partial assignments obtained in E-step.
	\item Remember that EM is a clustering algorithm. For example, we assume the data is drawn from $K$ normal distributions, and our job is to partially classify the data into the distributions. 
	\item What EM is doing is to find the MLE for parameters 
	\item
	\item
\end{enumerate}

\subsection{Sampling}
\begin{enumerate}
	\item Gibbs Sampling: 
	\item
	\item
	\item
	\item
	\item
\end{enumerate}

\subsection{}
\begin{enumerate}
	\item 
	\item
	\item
	\item
	\item
	\item
\end{enumerate}

\subsection{Topic models}
\begin{enumerate}
\item 
\item
\item
\item
\item
\item
\end{enumerate}

\subsection{Popular library/software for ML}
\begin{enumerate}
	\item Tensor flow
	\begin{enumerate}
		\item An open-source software library for Machine learning
		\item Developed by Google
		\item TensorFlow computations are expressed as stateful dataflow graphs. The name TensorFlow derives from the operations which such neural networks perform on multidimensional data arrays. These multidimensional arrays are referred to as "tensors". In June 2016, Google's Jeff Dean stated that 1,500 repositories on GitHub mentioned TensorFlow, of which only 5 were from Google.
		\item 
		\item
		\item
	\end{enumerate}
	\item Apache Hadoop
	\begin{enumerate}
		\item For distributed computing. 
		\item Allows for the distributed processing of large datasets across clusters of computers using simple programming models. 
		\item To scale up from single serves to thousands of machines, each offering local computation and storage.
		\item
		\item
		\item
	\end{enumerate}
	\item MapReduce
	\begin{enumerate}
		\item The term MapReduce actually refers to two separate and distinct tasks that Hadoop programs perform. The first is the map job, which takes a set of data and converts it into another set of data, where individual elements are broken down into tuples (key/value pairs). The reduce job takes the output from a map as input and combines those data tuples into a smaller set of tuples. As the sequence of the name MapReduce implies, the reduce job is always performed after the map job.
		\begin{figure*}[tb] %[tb]
%			\squeezeup
			\centering%width 8.5 cm
			\includegraphics[width=0.8\textwidth]{MapReduce.PNG}
			%	height=2.2cm
			%\caption{Decisions by \SS for any item $\w$.}
			%\squeezeup\squeezeup
			\label{fig:potentialVertex}
		\end{figure*}
		\item
		\item
		\item
		\item
		\item
	\end{enumerate}
	\item Apache Spark
	\begin{enumerate}
		\item 
		\item
		\item
		\item
		\item
		\item
	\end{enumerate}
	\item
	\item
\end{enumerate}


\end{multicols}
\end{document}