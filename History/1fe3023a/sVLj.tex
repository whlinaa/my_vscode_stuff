\documentclass[11pt, oneside]{article}   
\usepackage{C:/Users/whlin/Documents/Latex/Styles/My_Style/Assignment}
% If the doc. is for printing, use the segment to change all colors to black
\begin{comment}
\hypersetup{colorlinks, citecolor=black, filecolor=black, linkcolor=black, urlcolor=black, pdftex}
\end{comment}

\title{Additional Tutorial Exercise 1}
\author{Wing-Ho Lin}
\date{}

\parindent 2em %\linespread{1.3}


\begin{document}
\maketitle 
%\tableofcontents  %show table of content
%\newpage
%\begin{multicols*}{2}  % the "*" means the lengths of the two columns can differ. note that additional "[]" can be used to make introductory paragargh to the two columns 

\section{Logic}

Let $P(n)$ be the proposition ``$(ab)^n=a^nb^n$", where $n$ is any positive integer.\newline
Basis Step: We show that $P(1)$ is true. When $n=1$, $\text{LHS}=(ab)^1=ab$. $\text{RHS}=a^1b^1=ab$. So $P(1)$ is true.\newline
Inductive step. Assume $P(k)$ is true, where $k$ is any positive integer. That is, $(ab)^k=a^kb^k$. We show that $P(k+1)$ also true. When $n=k+1$, we have $\text{LHS}=(ab)^{k+1}=(ab)^k\cdot(ab)=a^kb^k(ab)=a^{k+1}b^{k+1}=\text{RHS}$. Therefore, $P(k+1)$ is also true. By the principle of MI, we have proved that $P(n)$ is true for all positive integer.


$
\begin{bmatrix}
	0 & 0 & 0 & 0  & 1 & 1 & 1 & 0 \\
	0 & 0 & 0 & 0  & 1 & 1 & 0 & 1 \\
	0 & 0 & 0 & 0  & 0 & 1 & 0 & 1 \\
	0 & 0 & 0 & 0  & 0 & 0 & 1 & 1 \\
	1 & 1 & 0 & 0  & 0 & 0 & 0 & 0 \\
	1 & 1 & 1 & 0  & 0 & 0 & 0 & 0 \\
	1 & 0 & 0 & 1  & 0 & 0 & 0 & 0 \\
	0 & 1 & 1 & 1  & 0 & 0 & 0 & 0 
\end{bmatrix}
$

$
\begin{array}{l}
u'_0 = {u_0} + u_1^Tt + {\textstyle{1 \over 2}}{t^T}{U_2}t\\
{u'_1} = {u_1} + {U_2}t\\
{U'_2} = {U_2}
\end{array} 
$


Assume $m$ is even. Then, $m=2q$, where $q\in\Z$. So, $m^3=(2q)^3=8q^3=2(4q^3).$ Now, because $q\in\Z$, so $4q^3\in\Z$. Therefore, $m^3$ is even.


Assume that $m$ is odd and $m^3$ is even. 
Then $m=2q+1$, where $q\in\Z$. Then we have $m^3=(2q+1)^3=8q^3+12q^2+6q+1=2(4q^3+6q^2+3q)+1$.
Now, because $q\in\Z$, so $4q^3+6q^2+3q\in\Z$. So, $m^3$ is odd, contradicting with our assumption that $m^3$ is even.
So, if $m$ is odd, then $m^3$ is also odd.






$
\begin{array}{l}
m^3=(2q)^3\
\end{array} 
$

$
x\in (A\cup B)-\bar{C} \\
\Leftrightarrow x\in A\cup B ~\wedge~ x\notin\bar{C} \qquad\text{(Definition)}\\
\Leftrightarrow (x\in A \vee x\in B) ~\wedge~ x\notin\bar{C} \qquad\text{(Definition)}\\
\Leftrightarrow  (x\in A \vee x\in B) ~\wedge~ x\in C \qquad\text{(Definition)}\\
\Leftrightarrow x\in C ~\wedge~ (x\in A ~\vee~  x\in B)  \qquad\text{(Commutative)}\\
\Leftrightarrow (x\in C ~\wedge~ x\in A) ~\vee~ (x\in C ~\wedge~ x\in B) \qquad\text{(Distribution)}\\
\Leftrightarrow (x\in A ~\wedge~ x\in C) ~\vee~ (x\in B ~\wedge~ x\in C) \qquad\text{(Commutative)}\\
\Leftrightarrow x\in A\cap C ~\vee~ x\in B\cap C \qquad\text{(Defintion)}\\
\Leftrightarrow x\in (A\cap C) ~\cup~ (B\cap C) \qquad\text{(Definition)}\\
$


Yes, the function is one-to-one. Suppose that $f(x)=f(y)$, where $x,y\in\Z$. Then,
$
f(x)=f(y)
\Rightarrow \lfloor 3x-1\rfloor=\lfloor3y-1\rfloor
\Rightarrow 3x-1=3y-1 
\Rightarrow x=y
$

Let $y\in\Z$. We need to find $n\in\Z$ such that $f(n)=y$. Now,
$f(n)=y\Rightarrow 3n-1=y\Rightarrow 3n-1=y\Rightarrow n=(y+1)/3$. 



\newpage

Let $f:\Z^+\rightarrow\Z^+$ be defined by $f(n)=n+1$. Which of the following is TRUE?

$P(A\cup B)\subseteq P(A)\cup P(B)$ 

$P(A)\cup P(B)\subseteq P(A\cup B)$ 

$P(A\cap B)\subseteq P(A)\cap P(B)$ 

$P(A)\cap P(B) \subseteq P(A\cap B)$

$2+6+10+\cdots+2(2n-1)=2n^2+2$


\begin{question}[Unreliable teachers]
Suppose there are two types of teachers. One type is \textbf{knight}, who always speaks truly; another type is \textbf{knave}, who always speaks falsely. There are two teachers A and B, each of whom is either a knight or knave. Answer each of the following questions.
\begin{itemize}
	\item[(a)] A makes the following statement: ``At least one of us is a knave." What are A and B?
	\item[(b)] Suppose A says, ``Either I am a knave or B is a knight." What are A and B?
	\item[(c)] Suppose A says, ``I am a knave, but B isn't." What are A and B?
	\item[(d)] Suppose now that we have three teachers A, B, C, each of whom is either a knight or a knave. A and B make the following statements:
	\begin{enumerate}
		\item A: All of us are knaves.
		\item B: Exactly one of us is a knight.
	\end{enumerate}
What are A, B,and C?
\end{itemize}
\end{question}
\begin{question}[Who are the robbers?]
 A bank was robbed. The police has found three suspects, A,
	B and C, and the following information:
	\begin{enumerate}
		\item A won't rob alone.
		\item	If C robs, he will rob with B.
		\item	If A doesn't rob, then C won't rob.
		\item	Among A and C, at least one of them robbed.
	\end{enumerate}
	Who robbed the bank?
\end{question}
\begin{question}[Who grinded her teeth?]
	Four girls, A, B, C and D, went on a trip to Europe. On
	the first day, they slept in the same room. In the midnight,
	D was waked up by someone who grinded her teeth.
	Next day, D found out the following information:
	\begin{enumerate}
		\item A grinded her teeth if and only if C grinded her teeth.
		\item  Not both A and B grinded their teeth.
		\item If C didn't grind her teeth, then A did.
	\end{enumerate}
Who grinded her teeth?
\end{question}

\begin{question}Show that the following argument is valid.
	\begin{enumerate}
		\item $(A\rightarrow B)\rightarrow C$
		\item $\neg D\vee A$
		\item $\neg D \rightarrow (A\rightarrow B)$
		\item $\neg A$~~$/ \therefore C$
	\end{enumerate}
\end{question}

\begin{question}Show that the following argument is valid.
	\begin{enumerate}
		\item $(A\rightarrow B)$
		\item $(C\rightarrow D)$
		\item $(B\vee D)\rightarrow E$
		\item $\neg E~~$/ $\therefore \neg(A\vee C)$
	\end{enumerate}
\end{question}
\begin{question}
	Prove the following logical equivalences using equivalence rules.
	\begin{enumerate}
		\item[(a)] $(p\rightarrow q)\wedge(p\rightarrow r)\equiv p\rightarrow(q\wedge r)$
		\item[(b)] $(p\rightarrow q)\vee(p\rightarrow r)\equiv p\rightarrow(q\vee r)$
		\item[(c)] $(p\rightarrow r)\wedge(q\rightarrow r)\equiv (p\vee q)\rightarrow r$
		\item[(d)] $(p\rightarrow r)\vee(q\rightarrow r)\equiv (p\wedge q)\rightarrow r$
	\end{enumerate}
\end{question}

\begin{question}[Distribution for XOR?]
	Are the following ``distribution rules" for XOR correct?	Prove or disprove:
	\begin{enumerate}
		\item[(a)] $p\wedge(q\oplus r)\equiv(p\wedge q)\oplus(p\wedge r)$
		\item[(b)] $p\vee(q\oplus r)\equiv(p\vee q)\oplus(p\vee r)$
	\end{enumerate}
\end{question}

%\begin{question}[Associativity for implication?]
%	Are the following ``Associative rule" for implication (i.e., $\rightarrow$) correct?	Prove or disprove:
	%\begin{enumerate}
%	$$(p\rightarrow q) \rightarrow r\equiv p\rightarrow (q\rightarrow r)$$
	%\end{enumerate}
%\end{question}

\begin{question}
	Show that each of the followings is a tautology:
	\begin{itemize}
		\item[(a)] $(\neg q\wedge (p\rightarrow q))\rightarrow\neg p$
		\item[(b)] $(p\wedge q)\rightarrow(p\vee q)$
	\end{itemize}
\end{question}



\section{Proof}
\begin{question}
	Prove the triangle inequality, which states that if $x$ and
	$y$ are real numbers, then $\abs{x} + \abs{y} \ge \abs{x + y}$.
\end{question}
\begin{question}
	Use a proof by contradiction to show that there is no rational
	number $r$ for which $r^3 + r + 1 = 0.$ [Hint: Assume
	that $r = a/b$ is a root, where $a$ and $b$ are integers and $a/b$
	is in lowest terms. Obtain an equation involving integers
	by multiplying by $b^3$. Then look at whether $a$ and $b$ are
	each odd or even.]
\end{question}
\begin{question}
	Prove that at least one of the real numbers $a_1, a_2,\dots,a_n$ is greater than or equal to the average of these numbers.
\end{question}
\begin{question}
	Use a proof by contraposition to show that if $x + y \ge 2$, where $x$ and $y$ are real numbers, then $x \ge 1$ or $y \ge 1$.
\end{question}
\begin{question}
	Prove or disprove that the product of two irrational numbers is irrational.
\end{question}
\begin{question}
	Prove or disprove that the product of a nonzero rational number and an irrational number is irrational.
\end{question}
\begin{question}
	Prove by contraposition that if $(n+1)^2$ is odd, then $n+1$ is odd.
\end{question}
\begin{question}
		Prove by cases that $n^3-n$ is a multiple of 3 for all $n\in\N$.
\end{question}


\begin{itemize}
	\item $\{a,b\}\in\{a,b,c,\{a,b\}, R \}$
	\item $\{a,b\}\subseteq\{a,b,\{a,b\}, R\}$
	\item $\{a,b\}\subseteq P(\{a,b,\{a,b\},R\})$
	\item $\{\{a,b\}\}\in P(\{a,b,\{a,b\},R\})$
	\item $\{a,b,\{a,b\}\}-\{a,b\}=\{a,b\}$
	\item $A^{R+3}=A^{R+1}\times A^2$, where $A$ is a nonempty set.
	\item $\emptyset\in\emptyset\times A^R$, where $A$ is a nonempty set.
	\item $\emptyset=\emptyset\times \emptyset^R$
\end{itemize}


\begin{question}~
\begin{enumerate}
	\item $\emptyset=\emptyset\times \emptyset^{R+2}$ % T
	\item $\{\{\{c\}\}\}\subseteq P(\{b,\{c\}\})$ %T
	\item $\{a,b,R\}\in\{a,b,c,\{a,b\}, R \}$ %F
	\item $\{\{a,b\}\}\in P(\{a,b,\{a,b\},R\})$ %T
	\item $\{a,b,\{a,b\}, R\}-\{a,b\}\subseteq\{a,b,R\}$ %F
	\item $A^{R+5}=A^{R+3}\times A^2$, where $A$ is a nonempty set. %F
	\item $\emptyset\in\emptyset\times A^{R+1}$, where $A$ is a nonempty set.%F
	\item $\emptyset^R\subset \emptyset\times \emptyset^{R}\times\{\}$% F
\end{enumerate}
\end{question}

Hello world!



%\begin{question}
%\end{question}
%\end{multicols*}
\end{document}